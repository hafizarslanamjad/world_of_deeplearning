\section{Subtraction in Temporal and Logical Structures}
In time-series data, systems are modeled as sequences of values indexed over time. The expression $x_t - x_{t-1}$ computes the change between successive time steps. This operation removes prior state information, isolating what is new. From a logical standpoint, subtraction here acts as a form of \emph{contrastive reasoning}: it eliminates what is already known (the past state) and retains only what has changed (the present deviation).

This idea extends to structured symbolic representations such as graphs or knowledge embeddings. In logical anomaly detection frameworks like ComAD or SINBAD, each relation or node is encoded as a vector. Subtracting an expected relationship $f(B)$ from an observed one $f(A)$ gives:
\[
\Delta = f(A) - f(B)
\]
This vector $\Delta$ captures the semantic deviation. The subtraction removes commonalities and reveals contradiction or novelty. Thus, logical reasoning is achieved through algebraic difference.