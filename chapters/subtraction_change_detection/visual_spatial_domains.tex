\section{Subtraction in Visual and Spatial Domains}
In image processing, edges—points of abrupt intensity change—can be detected using filters that apply subtraction across spatially neighboring pixels. A common example is the Sobel filter used for horizontal edge detection:
\[
K = \begin{bmatrix}
	-1 & 0 & 1 \\
	-2 & 0 & 2 \\
	-1 & 0 & 1 \\
\end{bmatrix}
\]
This kernel applies the expression $(\text{right intensity}) - (\text{left intensity})$, highlighting areas with strong horizontal contrast. The center row is weighted more because real images often carry more structural information in the center.

Here, subtraction works locally but serves a global purpose: it marks boundaries and discontinuities. The logic is again contrastive—only differences survive the operation.