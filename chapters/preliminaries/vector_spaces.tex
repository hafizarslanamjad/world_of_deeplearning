\section{Understanding Closure in \(\mathbb{R}^n\): Why Vector Spaces Are Chosen for Image Representation}

A vector space is a mathematical structure consisting of a set of  vectors that support two key operations: vector addition and scalar multiplication. These operations must satisfy a list of specific axioms—such as associativity, distributivity, and most importantly for our discussion, \emph{closure}. The space \(\mathbb{R}^n\), for instance, is the set of all \(n\)-dimensional vectors whose components are real numbers:
\[
x = \begin{bmatrix} x_1 \\ x_2 \\ \cdots \\ x_n \end{bmatrix}, \quad \text{where each } x_i \in \mathbb{R}.
\]
This makes \(\mathbb{R}^n\) an example of a real vector space.

\subsection{Meaning of Closure}
Closure under addition and scalar multiplication means that for any two vectors \( u, v \in \mathbb{R}^n \), their sum \( u + v \in \mathbb{R}^n \), and for any scalar \( a \in \mathbb{R} \), the scaled vector \( a u \in \mathbb{R}^n \). 

\subsection{Why Closure Matters for Images}
When an image is represented as a vector (for instance, flattening pixel intensities into an \(n\)-dimensional vector), we often wish to apply operations such as blending two images or adjusting brightness. These correspond respectively to vector addition and scalar multiplication. To ensure the result of these operations remains a valid image, we need the underlying representation space to be closed under these operations. Vector spaces like \(\mathbb{R}^n\) give us this guarantee by design.

\subsection{How \(\mathbb{R}^n\) Ensures Closure}
The structure of \(\mathbb{R}^n\) is built component-wise from the real numbers, and real numbers themselves are closed under addition and multiplication. That is:
\[
(x_1, \dots, x_n) + (y_1, \dots, y_n) = (x_1 + y_1, \dots, x_n + y_n),
\]
\[
a \cdot (x_1, \dots, x_n) = (a x_1, \dots, a x_n),
\]
where each operation is closed within \(\mathbb{R}\). Therefore, every result remains inside \(\mathbb{R}^n\).

Thus, when we say that \emph{“we are choosing to represent images in a space that is already structured to ensure closure,”} we mean that the vector space \(\mathbb{R}^n\) is not arbitrarily chosen. It is selected because it is algebraically and geometrically equipped with properties that make mathematical operations on images consistent, safe, and computationally tractable. Closure is one of the fundamental pillars of this structure.

\section{Understanding Column Space and Its Relevance in Deep Learning}

Linear algebra introduces the notion of a \textit{vector space} as a fundamental structure in which vectors can be added and scaled while remaining within the same space. Once, one understands that a space is defined by its basis vectors, the next conceptual step is to understand how specific types of subspaces emerge from matrices. One such subspace, central to both theoretical and applied linear algebra, is the \textit{column space}.

The term ``column space'' arises from a natural and practical consideration. When a matrix $A$ acts on a vector $x$ through multiplication ($Ax$), the result is a linear combination of the columns of $A$. That is, each element in the vector $x$ serves as a coefficient that scales one of the columns of $A$, and the final output is the sum of these scaled columns. Therefore, the set of all possible outputs that can be generated by multiplying $A$ with any vector $x$ forms a space—specifically, the \textit{space spanned by the columns of $A$}. This is precisely why it is called the column space: it is the image (or range) of the transformation defined by $A$, and that image is built solely from combinations of its columns.

The concept of column space is not merely a theoretical construction; it plays a pivotal role in determining the behavior and capacity of models in machine learning, particularly in deep learning. Consider a linear transformation implemented by a weight matrix $W$ within a neural network layer. The output of this layer, for any input vector $x$, is $Wx$, which lies entirely within the column space of $W$. No matter how complex the input data is, the transformation cannot produce an output vector outside of the column space. This reveals a critical limitation: if the column space is low-dimensional due to dependent columns (i.e., $W$ has low rank), then the layer has a limited capacity to represent diverse outputs. Such a situation can lead to underfitting, where the model fails to capture the complexity of the data.

Moreover, when designing autoencoders or dimensionality-reducing architectures such as bottleneck networks, the encoder's weight matrix defines a lower-dimensional subspace where input data is projected. This subspace is none other than the column space of the encoder matrix. The success of such models depends on how well this column space captures the essential structure of the input data. If the column space misses important directions, then crucial information is irretrievably lost during encoding.

Even in unsupervised methods like Principal Component Analysis (PCA), the reduced representation of data is achieved by projecting it onto the column space of the principal component matrix. Thus, the quality of dimensionality reduction is inherently tied to how well the column space aligns with the intrinsic geometry of the data distribution.

Furthermore, during the initialization of deep networks, care is taken to ensure that the initial weight matrices span a sufficiently rich column space. Initialization methods such as Xavier or orthogonal initialization aim to create column spaces that are isotropic and well-distributed, allowing gradients to propagate effectively and models to explore a wide range of outputs early in training.

In conclusion, while the term ``space'' might initially evoke an abstract mathematical landscape, the specification of ``column space'' grounds this abstraction in the practical mechanics of linear transformations. It denotes the exact set of outputs a matrix can produce and, in doing so, highlights the expressive boundaries of models in deep learning. Understanding column space equips one with the tools to reason about capacity, compression, and learnability in neural architectures.